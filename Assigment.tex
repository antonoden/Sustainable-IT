
Sida
1
av 2
\documentclass[conference,a4paper]{IEEEtran}
\usepackage{graphicx,caption,subfig,amsmath,array}
\usepackage[backend=biber]{biblatex}
\addbibresource{refs.bib}
\begin{document}
\title{My Paper}
\author{
\IEEEauthorblockN{Author 1, Author 2}
\IEEEauthorblockA{Dept. of Maths and Computer Science\\Karlstad University\\
65188 KARLSTAD, Sweden}
}
\maketitle
\begin{abstract}
Skriv detta kapitel absolut sist. Det är en sammanfattning av hela uppsatsen,
där varje del berörs. Personer som läser hundratals uppsatser i veckan begränsar
sig ofta till abstraktet. Detta är er akademiska "sales pitch". Max 200 ord och ska
inte innehålla referenser.
\end{abstract}
\section{Introduction}
Efter att ha läst introduktionen, bör läsaren förstå varför det vetenskapliga
bidraget i rapporten är intressant och aktuellt. Börja brett och smalna av mot ditt
område och ditt bidrag. Introduktionen ska svara på varför läsaren överhuvudtaget
ska läsa den, varför den är relevant, och varför området är viktigt. Början av
introduktionen ska vem som helst kunna läsa och hänga med i, men allt eftersom ditt
bidrag presenteras, ökar kunskapskraven.
\section{Bakgrund}
Läser man enbart om man inte forskar inom ämnet som läsare. Redogör för andra
forskares arbete, och presenterar den bakgrundsinformation som krävs för att förstå
resten av rapporten. Här ska det inte vara med några resultat, någon analys eller
några egna reflektioner.
\section{Analys}
Först här får ditt ert bidrag och era egna reflektioner vara med. Förkunskaper och
bakgrundsinformation ska däremot vara i bakgrundskapitlet.
\section{Slutsats}
Vad för slutsatser kan ni dra från ert arbete? Om läsaren hoppar direkt från
abstraktet eller introduktionen till slutsatsen, så ska den röda tråden i rapporten
vara välbehållen, och texten vara förståelig.
\section{Exempel}
Vanlig text med lite~\cite{biggs84} referenser~\cite{web:pox}. Innehåller även \
texttt{texttt}, \emph{emph}, \textit{kursiv} och \textbf{fet}. Dessutom innehåller
den lite ``citationstecken'' och punktlistor.
\begin{itemize}
\item Punktlista 1.
\item Punktlista 2.
\end{itemize}
\begin{enumerate}
\item Nummerlista 1.
\item Nummerlista 2.
\end{enumerate}
\begin{description}
\item[Test 1] Item 1.
\item[Test 2] Item 2.
\end{description}
\subsection{En underrubrik}
Med Figur~\ref{fig1}
\begin{figure}
\center
\includegraphics[width=0.4\textwidth]{plot1.png}
\caption{Figurtext}
\label{fig1}
\end{figure}
\printbibliography
\end{document}
