\documentclass[conference,a4paper]{IEEEtran}
\usepackage{graphicx,caption,subfig,amsmath,array}
\usepackage[backend=biber]{biblatex}
\usepackage{lipsum}
\addbibresource{refs.bib}
\begin{document}
\title{My Paper}
\author{
\IEEEauthorblockN{Anton Odén}
\IEEEauthorblockA{Dept. of Maths and Computer Science\\Karlstad University\\
65188 KARLSTAD, Sweden}
}
\maketitle
\begin{abstract}
Skriv detta kapitel absolut sist. Det är en sammanfattning av hela uppsatsen,
där varje del berörs. Personer som läser hundratals uppsatser i veckan begränsar
sig ofta till abstraktet. Detta är er akademiska "sales pitch". Max 200 ord och ska
inte innehålla referenser
\end{abstract}
\section{Introduction}
Efter att ha läst introduktionen, bör läsaren förstå varför det vetenskapliga
bidraget i rapporten är intressant och aktuellt. Börja brett och smalna av mot ditt
område och ditt bidrag. Introduktionen ska svara på varför läsaren överhuvudtaget
ska läsa den, varför den är relevant, och varför området är viktigt. Början av
introduktionen ska vem som helst kunna läsa och hänga med i, men allt eftersom ditt
bidrag presenteras, ökar kunskapskraven.
\section{Bakgrund}
Ett datacenter är en mängd servrar som i sig är en mängd processorkraft. När dessa processorer
utför det arbete som servrarna är uppsatta för så skapar de värme. Denna värme kan när den uppnår
nog höga temperaturer skada komponenter i processorn och omkringliggande komponenter. Värmen kan 
när den blir för hög skapa förutsättningarna för eld. Elden är i sin tur eld. Dåligt.

För att transportera bort den värme som skapad i processorerna behövs ett kylsystem. Till och börja med 
ligger en kylpasta ovanför CPUn. Ovanför pastan sitter en kylfläns monterad som är konstruerad så att det
ska finnas så stor yta som möjligt för värme att extraheras. Kylflänsen är gjord i metall som är ett bra 
ämne för att trasnportera värme. 

Efter kylflänsen har vi en monterad fläkt som 
Sen så har vi olika system för att föra fram kyla.  
\begin{enumerate}
    \item Det vanligaste systemet jobbar med tillförande av kyld luft.
    \item Sekundära i topplistan är att tillföra vatten i en kylande temperaturer
    \item Det rekommenderade för effektivitet är att tillföra kyld 2 komponent vätska (AC-medel)
\end{enumerate}

För att kyla kyla ner de ovan nämnda element så att dessa kan kyla ned CPUn behövs en kylningsenhet 
och transport mellan CPUn och denna kylenhet (CRAC). Kylenheten kan i sin tur ha olika teknologier för
att kyla ner elementen. 
    \section{Analys}
Anledningen till att en majoritet av datacenter väljer att gå på luftkylning är för att det är billigt och 
"enkelt". Dock så finns det många olika sätt att göra luftkylningen effektivare och dyrare vilket gör att det 
bör finnas nån punkt där det blir billigare med vattenkylning eller tvåkomponentskylning. Eftersom effekten ska
vara bättre med dessa kylelement så ska energiåtgången bli lägre på årsbasis vilket också ska räknas in i kalkylen. 
\section{Slutsats}
Vi får se vad jag kommer fram till efter min läsning och mina reflektioner. 
\section{Exempel}
Vanlig text med lite~\cite{energy3} referenser~\cite{web:pox}. Innehåller även \
texttt{texttt}, \emph{emph}, \textit{kursiv} och \textbf{fet}. Dessutom innehåller
den lite ``citationstecken'' och punktlistor.

\begin{enumerate}
\item Nummerlista 1.
\item Nummerlista 2.
\end{enumerate}
\begin{description}
\item[Test 1] Item 1.
\item[Test 2] Item 2.
\end{description}
\subsection{En underrubrik}
Med Figur
\lipsum[1]
\printbibliography
\end{document}

