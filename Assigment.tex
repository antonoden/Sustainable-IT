\documentclass[conference,a4paper]{IEEEtran}
\usepackage{graphicx,caption,subfig,amsmath,array}
\usepackage[backend=biber]{biblatex}
\usepackage{lipsum}
\addbibresource{refs.bib}
\begin{document}
\title{My Paper}
\author{
\IEEEauthorblockN{Anton Odén}
\IEEEauthorblockA{Dept. of Maths and Computer Science\\Karlstad University\\
65188 KARLSTAD, Sweden}
}
\maketitle
\begin{abstract}
Skriv detta kapitel absolut sist. Det är en sammanfattning av hela uppsatsen,
där varje del berörs. Personer som läser hundratals uppsatser i veckan begränsar
sig ofta till abstraktet. Detta är er akademiska "sales pitch". Max 200 ord och ska
inte innehålla referenser
\end{abstract}
\section{Introduction}
Efter att ha läst introduktionen, bör läsaren förstå varför det vetenskapliga
bidraget i rapporten är intressant och aktuellt. Börja brett och smalna av mot ditt
område och ditt bidrag. Introduktionen ska svara på varför läsaren överhuvudtaget
ska läsa den, varför den är relevant, och varför området är viktigt. Början av
introduktionen ska vem som helst kunna läsa och hänga med i, men allt eftersom ditt
bidrag presenteras, ökar kunskapskraven.
\section{Bakgrund}
Ett datacenter är en mängd servrar som i sig är en mängd processorer och hårddiskar. 
När dessa komponenter utför det arbete som användaren utsätter dom för skapar de värme. Denna värme kan när 
den uppnår nog höga temperaturer skada komponenter i servrarna. Värmen kan när den blir nog hög till och med
skapa förutsättningarna för eld. referens?. Vilket skapar ytterligare ekonomisk skada, men kan även bli en 
fara för liv. 

För att transportera bort den värme som skapas i servrarna behövs ett kylsystem. Kylsystemen kan till 
och börja med delas upp i 3 olika kategorier. Kategorierna delas upp baserat på vilket tillstånd, 
flytande eller gas, som det kylande ämnet är i. 

\begin{enumerate}
    \item Luftkylning
    \item Flytande kylning 
    \item Tvåfaskylning
\end{enumerate}

Det vanligaste sättet att kyla datacenter idag är via luftkylning. Det går till så att kyld luft 
transporteras genom servrarna tack vare fläktar och gärna tunnelsystem för bättre riktad kylning. 
Benämning på den anläggning som sköter hanteringen av luften är CRAH (Computer Room Air Handling) \cite{modelling1}.
Idealet är att konstruera ett datacenter där luften transporteras bort den väg vi vill att den ska ta. 
Då det blir ett problem när den uppvärmda luften blandas med den kylda luften, då blir inte effekten
av den kylda luften lika god. Därför finns det många olika nivåer på hur ett datacenter konstrueras 
för att se till att den uppvärmda luften transporteras bort från servrarna. \cite{modelling2} \cite{}

Det andra sättet är att kyla datacentret är med ett flytande element. Det flytande elementet kan vara 
vatten men det kan också vara ett antistatiskt ämne som komponenterna inte tar skada av om det skulle 
få kontakt. Den flytande kylningen kan via rörledningar föras tätt intill de delar i servrarna som
skapar mest värme. Och sen via rör ledas bort från servrarna. Använts ett antistatiskt ämne så kan 
hela servrarna sänkas ner i ett bad för maxial kylning. 

Tvåfaskylning heter så för att det ämne som används för kylningen ändrar tillstånd från flytande till gas 
under kylning då det har en koktemperatur som understiger temperaturen på de komponenter som ska kylas. 
I gasform så transporteras ämnet snabbare bort till en anläggning som kyle ner gasen igen och som sen 
transporteras ämnet åter igen till servrarna. En stor fördel med tvåkomponent är att systemet i sig skapar ett 
högre tryck vilket gör att det är mindre beroende av pumpar än ett endast flytande system. referens?

Det finns att antal komponenter som kan behövas i en kylanläggning. Lite beroende på vilket metod som valts 
att bygga upp kylanläggning i. 
(Computer Room Air Handling) CRAH (ström, investering)
pumpar (ström, investering)
fläktar (ström, investering)
rör (investering)
Luftströmningstrummor (investering) 
(Computer Room Air Chiller) CRAC (ström, investering)

\[P_{DC} = P_{IT} + P_{strom} + P_{kylning} + P_{ljus} \]

Strömförbrukningen för ett datacenter kan delas upp i olika delar. Själva strömförbrukningen av servrarna 
kan benämnas P\underline{IT}. Strömförbukningen av kylningsanläggningen kan benämnas P\underline{kylning}. 
Förlorad ström i distribueringssystemet för ström kan benämnas P\underline{strom}. Sen så behöves lokaler
där datacentret kan placeras och dessa vill vi gärna göra människovänliga med ljus och övrig teknik. Den 
förbrukningen kan benämnas P\underline{ljus}. 

För att jämföra datacentrets strömförbrukning med fokus de de komponenter som inte har med servrarnas arbete
att göra så kan vi dela alla dessa förbrukningar med P\underline{IT}. Då får vi ut ett PUE-värde (Power usage effectiveness) \cite{modelling1}

\[PUE = (P_{IT} + P_{strom} + P_{kylning} + P_{ljus}) / P_{IT} \]

Det idealiska är att PUE-värdet blir 1. Då är det endast servrarna som använder ström. Eftersom kylningen står för 29-50\%
av datacentrets strömförbrukning [referens] så är det den som ger mest effekt att sänka förbrukningen på.  

Det finns också en rörelse att återvinna den värme som skapas så att vi använder energin till mer än datorkraft. 
Detta kan som uppe i Boden innebära att värmen transporteras till ett intilligande växthus där den används för 
att skapa en väztzon uppe i kallaste norden där grönsaker kan odlas året om. Förespråkare av återvinnande av 
värmeenergin vill lägga till P\underline{atervinn} i PUE ekvationen. P\underline{atervinn} är den mängden ström 
som skulle ha behövts användas för att värma upp växthuset till den grad som datacentret nu gör med sin värme.

\[PUE = (P_{IT} + P_{strom} + P_{kylning} + P_{ljus} - P_[atervinn]) / P_{IT} \]

Att tänka på vid användandet av återvinningsförbruknigen är att vi kan ha väldigt inneffektiva datacenter som visar
bra värden eftersom P\underline{atervinn} blir stort.  Det är då viktigt att tänka på om växthuset kunnat värmas med 
en ström/energikälla som hade varit mer miljövänlig än den strömmen som matas in i datacentret. Placeringen av växthuset
bör vara på en sådan plats så att den extra värmen kan utnyttjas större delen av året. Annars har investeingen att koppla
ihop de olika verksamheterna kostat både ekonomiskt och miljömässigt mer än det smakat. 

\section{Analys}
Anledningen till att en majoritet av datacenter väljer att gå på luftkylning är för att det är billigt och 
"enkelt". Dock så finns det många olika sätt att göra luftkylningen effektivare och dyrare vilket gör att det 
bör finnas nån punkt där det blir billigare med vattenkylning eller tvåkomponentskylning. Eftersom effekten ska
vara bättre med dessa kylelement så ska energiåtgången bli lägre på årsbasis vilket också ska räknas in i kalkylen. 
\section{Slutsats}
Vi får se vad jag kommer fram till efter min läsning och mina reflektioner. 
\section{Exempel}
Vanlig text med lite~\cite{energy3}. Innehåller även \
texttt{texttt}, \emph{emph}, \textit{kursiv} och \textbf{fet}. Dessutom innehåller
den lite ``citationstecken'' och punktlistor.

\printbibliography
\end{document}

