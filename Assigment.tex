\documentclass[conference,a4paper]{IEEEtran}
\usepackage{graphicx,caption,subfig,amsmath,array}
\usepackage[backend=biber]{biblatex}
\usepackage{lipsum}
\addbibresource{refs.bib}
\begin{document}
\title{Datacentrets kylsystem}
\author{
\IEEEauthorblockN{Anton Odén}
\IEEEauthorblockA{Dept. of Maths and Computer Science\\Karlstad University\\
651 88 KARLSTAD, Sweden}
}
\maketitle
\begin{abstract}
Datacenter är en industri som vi skapat genom bland annat vår skärmkonsumtion. En industri
som slukar elektrisk energi och som också gör detta mycket pga. ineffektiva uppsättningar.
Det är värdefull energi som skulle behövas i vår omställning från fossila bränslen. Att effektivisera våra datacenter
är en klimatåtgärd. Ett första steg är att förstå vilka alternativ som finns. Jag hoppas denna artikel
ska hjälpa läsaren med detta.  
\end{abstract}
\section{Introduction}
Jorden är en planet där vi människor gör oss allt mer beroende av digital hjälpmedel. 
Utvecklingen har skett på mindre än en mansålder och det har ökat exponentiellt. Det är
egentligen väldigt svårt att sia i framtiden. Hur mycket mer kommer vi använda oss av 
det digitala 2030, 2050? För att det digital systemet ska fungera som vi förväntar oss så är vi
beroende av datacenter. Datacenter skulle kunna vara en server i ditt hem. Men utvecklingen
går mot centralisering och större datakluster. I datacentret jobbar processorn dag och natt
för att hjälpa dess användare. Denna hjälp kostar ström. Och den skapar värme. Värmen kan 
skada processorn ifall den inte transporteras bort. Själva systemet för att transportera bort
denna värme är en stor del av strömförbrukningen i ett datacenter. Upp till 40\%. Eftersom 
alla datacenter i jordens, konsumerar upp till 2\% av jordens elproduktion så blir effektiviseringen 
av datacenters kylning viktig. 
\section{Bakgrund}
Ett datacenter är en mängd servrar som i sig är en mängd processenheter (CPU,GPU etc.), minnen och hjälpkomponenter. 
När framförallt CPUn utför det arbete som användaren utsätter den för, så skapar den värme. Denna värme kan när 
den uppnår nog höga temperaturer skada komponenter i servrarna. Till och med skapa förutsättningar för eld. 
För att transportera bort den värme som skapas i servrarna behövs ett kylsystem. Kylsystemen kan till 
och börja med delas upp i 3 olika kategorier. Kategorierna delas upp baserat på vilket tillstånd, 
flytande eller gas, som det kylande ämnet är i. 
\begin{enumerate}
    \item Luftkylning
    \item Flytande kylning 
    \item Tvåfaskylning
\end{enumerate}
\subsubsection{Luftkylning}
Det vanligaste sättet att kyla datacenter är via luftkylning. Det går till så att kyld luft 
riktas genom servrarna tack vare fläktar. Det nu uppvärmda luften behöver tas om hand så att den ej kontinaminerar 
den kylda luften. Detta görs genom fläkt och tunnelsystem. \cite{modelling2}\cite{modelling1}.
\subsubsection{Flytande kylning}
Ett annat sätt att kyla datacentret är med ett flytande ämne. Ett vanligt flytande ämne är vatten. Men 
också andra kylmedel så som t.ex HFC134a. \cite{energycompare2} Den flytande kylningen kan via ledningar föras tätt intill de delar i servrarna som
skapar mest värme och sen via andra ledningar ledas bort från servrarna. En flytande vätska blir istället för luft beroende av pumpar och bibehållet tryck i ledningsystemet. 
Använts ett antistatiskt kylmedel så kan hela servrarna sänkas ner i ett bad av kylmedel och vi blir inte lika beroende av pumpar.   
\subsubsection{Tvåfaskylning}
Tvåfaskylning heter så för att det ämne som används för kylningen ändrar tillstånd från flytande till gas 
under kylning då det har en gastemperatur som understiger temperaturen på de komponenter som ska kylas. 
I gasform så transporteras kylmedlet till en kondensator som kyler ner gasen igen och sen pumpas samma kylmedel tillbaka igen. 
\subsection{Gratis kylning}
På ovan förklarningar så har vi gått igenom slutna system där samma luft och flytande ämne återkyls och återanvänds 
i systemet. När de kommer till luft så kan vi ta in luft utifrån för att få kyld luft gratis. Detsamma gäller för vatten. 
I samma takt som luft eller vatten tillförs uteifrån. I samma takt behöver det också avlägsnas uppvärmd luft eller 
vatten från datacentret. 
\subsection{Strömförbukning}
För att få en matematisk förståelse över strömförbukningen så delar vi upp datacentrets olika strömförbrukare. 

\[P_{DC} = P_{IT} + P_{strom} + P_{kylning} + P_{ljus} \]

Själva strömförbrukningen av servrarna kan benämnas P\textsubscript{IT}. Strömförbukningen av kylningsanläggningen 
kan benämnas P\textsubscript{kylning}. Förlorad ström i distribueringssystemet för ström kan benämnas P\textsubscript{strom}. 
Övrig förbrukning för att hålla lokalerna människovänliga kan benämnas P\textsubscript{ljus}. 
\newline\newline
För att jämföra datacentrets strömförbrukning med fokus de de komponenter som inte har med servrarnas arbete
att göra så kan vi dela alla dessa förbrukningar med P\textsubscript{IT}. Då får vi ut ett PUE-värde (Power usage effectiveness) \cite{modelling1}

\[PUE = \frac{P_{IT} + P_{strom} + P_{kylning} + P_{ljus}}{P_{IT}} \]

Det idealiska är att PUE-värdet blir 1. Då är det endast servrarna som använder ström. Eftersom kylningen kan stå för 30-40\%
av datacentrets strömförbrukning så är det den som ger mest effekt att sänka förbrukningen på. \cite{modelling2}\cite{energy3}
\subsection{Återvunnen energi}
Det går också att återvinna den värme som skapas så att vi använder energin till mer än datorkraft. I kombination med gratis kylning
kan det göra datacentret klimatvänligare. Detta kan som uppe i Boden innebära att värmen transporteras till ett intilligande växthus. 
Där skapar den en växtzon uppe i kallaste norden, där grönsaker kan odlas året om.\cite{free-cooling1} Förespråkare av återvinnande av 
värmeenergin vill lägga till P\textsubscript{atervinn} i PUE ekvationen. P\textsubscript{atervinn} är den mängden ström 
som skulle ha behövts användas för att värma upp växthuset till den grad som datacentret nu gör med sin värme.

\[PUE = \frac{P_{IT} + P_{strom} + P_{kylning} + P_{ljus} - P_{atervinn}}{P_{IT}} \]  
\newline
\section{Analys}
Luftkylning är det mest utbredda sättet att transportera bort värme från servar. Detta för att det är billigast att installera
och underhålla. När ett system blir marknadsledande så som lutkylning så hjälper det också till att trycka ner priser. Dock vilket 
vi diskuterat innan så är det inte effektivt. Vid användning av enbart luftkylning så står kylningen för 30-50\% av datacentrets 
strömförbukning. 
\subsection{Luftkylning} 
Luftkylning är den vanligaste metoden för kylning och ser ut att fortsätta ha den positionen. Dess position som marknadsledande
ser ut att hotas av två saker. Strömförbrukningen av kylsystemet och att framtidens processorenheter blir allt mer energikrävande.
Vilket kommer skapa mer värme. En värme som endast luftkylning, ej klarar av att leda bort. 
\subsubsection{Strömförbrukningen}
Vid uppbyggnad av nya datacenter så blir det en kalkyl att göra. Hur mycket mer ström kommer det lyftkylda systemet 
förbruka jämfört med ett flytande eller tvåfassystem? Sedan se på vad investeringen och underhållet av de olika systemen kostar. 
Kostnaden för ström blir då en avgörande faktor. Vilket är en anledning till att svenska Norrland med sin relativt sett billiga el pga. 
Sveriges väl utbyggda vattenkraftssystem är en intessant plats att placera datacenter. 
\subsubsection{Framtidens processor}
Processenheter har blivit allt tätare och därmed kraftfullare sen start. Vilket resulterat i Moores lag. Vilken i korta ord säger 
att processortätheten/kraften dubbleras varannat år. Detta skapar allt större krav på att transportera bort värmen som blir större och 
större. Luftkylning har ett tak i hur mycket värme den kan transportera bort. Vilket gör att när processorkraften uppnår 1kW/cm2
så behövs annan kylning.
\subsection{Flytande kylning}
Flytande kylning är både ett komplement till luftkylningen men kan också vara en ersättare. Eftersom vi har med vätska och göra så 
kommer de tillkomma mer kostnader i underhåll för att säkerställa att vi inte får några läckor. Vid användning av vatten blir det extra
kritiskt då de elektriska komponenterna kan ta skada. En stor fördel med flytande kylning är att det transporterar värme längre sträckor vilket gör det
enklare att använda sig av den värme som skapas till andra ändamål. Används vatten kan den uppvärma vattnet kopplas in i närliggande lokalers
värmesystem och få en direkt påverkan på värmeelementen. Värmen skulle också kunna användas i skapandet av ny ström \cite{energycompare2}  
\subsection{Tvåfaskylning}
Denna form av kylning är fortsatt en nisch på marknaden. En stor fördel med tvåfaskylning dock är att det uppvärmda ämnet transporteras
iväg relativt naturligt när det går till gasform. Temperaturen på processorenheterna blir därmed relativt konstant efter vilket 
kokpunkt ämnet ligger på. Vi får därför en lägre strömkostnad för pumpar vilket är en stor kostnad för ett flytande system där det flytande
ämnet måste cirkuleras beroende på processorenhetens arbetstakt. 
\subsection{Gratis kylning}
Vid inhämtning av luft utifrån så behöver extra fokus läggas på luftfuktigheten. Även avfukting av luft kostar ström. Det är oftast så att 
utomhosfuktigheten är över rekommenderat 60\% relativ luftfuktighet för ett datacenter \cite{energy3}. Men ifall luftfuktigheten skulle lägre så kan 
det behövas adderas vatten till luften inann den tas in i datacentret. Det är viktigt att servrarna har en någerlunda konstant 
luftfuktighet. Vid låg luftfuktighet ökar risken för elektrisk urladdning vilket skadar komponenterna. 
\subsection{Återvunnen energi}
Att tänka på vid återvinning av energi är att det inte ska bli en form av green washing. Det kan väldigt lätt bli så att datacentret inte arbetar 
med att effektivisera sin förbrukningen eftersom den värme som skapas används i ett senare led. PUE talen om återvunnen energi tas med i beräkningen 
visar på väldigt bra värden. Datacentrets position blir viktig då Placeringen av växthuset bör vara på en sådan plats så att den extra värmen kan 
utnyttjas större delen av året. Annars har investeingen att koppla ihop de olika verksamheterna kostat både ekonomiskt och miljömässigt mer än det smakat.
\section{Slutsats}
I denna artikel har vi fått igenom luftkylning, flytande kylning och tvåfaskylning. Vi har sneglat på matematiska formler för att benchmarka ett datacenter. 
Vi har även tittat på hur vi kan använda oss av placeringen av ett datacenter och hur vi kan använda återanvända den värme som uppstår. Förhoppningsvis ska denna 
artikel ge läsaren en godare förståelse i hur ett datacenter kan kylas och att framtiden kräver ett mer mångsidigt användande av energin i ett datacenter. 

\printbibliography
\end{document}

