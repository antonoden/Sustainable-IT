
\documentclass[conference,a4paper]{IEEEtran}
\usepackage{graphicx,caption,subfig,amsmath,array,lipsum}
\usepackage[backend=biber]{biblatex}
\addbibresource{refs.bib}
\begin{document}
\title{Optimering av ett datacenters energiförbrukning, med fokus på kylsystemet}
\author{
\IEEEauthorblockN{Anton Odén, anton.oden@outlook.com}
\IEEEauthorblockA{Dept. of Maths and Computer Science\\Karlstad University\\
651 88 KARLSTAD, Sweden}
}
\maketitle
\begin{abstract}
Kylsystem kan vara en stor förbrukare av energi i ett datacenter. Luftkylning har fördelar av att vara marknadsmoget sen länge men det finns nackdelar 
i att endast förlita sig på denna teknik. Luftkylningen bör kombineras med gratis kylning, magasinering av vatten och en återvinningsplan på den värmeenergi 
som skapas. Vätskekylning är ett komplement som kan vara ett måste i högpresterande processorer då luft ej har tillräckligt stor värmeledningsförmåga, men 
det går även att ta det ett steg längre och sänka ner hela servrarna i kylarvätska. Huvudspåret bör alltid vara att få ett så effektivt utnyttjande 
av energi som möjligt.  
\end{abstract}
\section{Introduktion}
Människan gör sig allt mer beroende av digitala hjälpmedel. Utvecklingen har skett på mindre än en mansålder 
och det är egentligen väldigt svårt att förutspå hur beroende vi kommer vara av i framtiden.

För att det digitala systemet ska fungera som vi förväntar oss, finns ett behov av datacenter. Därför har antalet datacenter i Europa ökat med 5\% per år
mellan 2017 och 2022 ~\cite{marketoverview1}. Under åren 2020 till 2022 så påskyndandes utvecklingen tack vare Covid19-pandemin, som gjorde att många 
behövde kunna arbeta hemifrån, vilket ställde högre krav på fungerande system. Det senaste året har fokuset skiftat till AI, men vilken anledning det än 
är till att datacenter byggs ut, så sker det i en snabb takt. 

Datacenter behöver elektrisk energi för att operera. Operationerna omvandlar den elektriska energin till värmeenergi. För att servrarna ska fortsatt operera behöver 
värmeenergin transporteras bort, vilket kan ske på olika sätt. I ett luftkylt datacenter kan kylningen stå för 40\% av strömförbrukningen vilket ger PUE-värden närmare 2.0
~\cite{modelling2,energy3}. Modernt uppbyggda datacenter som exempelvis Metas datacenter i Luleå \cite{META}, presenterar PUE-värden så låga som 
1.09. Enligt dom själva är detta tack vare gratis kylning och tillförlitlig strömdistribution. Detta visar att planerad kylning i kombination med 
lokalisering av datacenter bidrar till en mer effektiv energianvändning. Uptime institue presenterade 2022 att det genomsnittliga rapporterade PUE-värdet låg på
1.55. \cite{PUE} 

\section{Bakgrund}
För att få en matematisk förståelse över ett datacenters strömförbrukning så delas de olika strömförbrukarna upp \cite{modelling2}. Strömförbrukningen 
av servrarna kan benämnas P\textsubscript{IT}, kylningsanläggningen P\textsubscript{kylning}, strömdistribuering P\textsubscript{strom} 
och övrig förbrukning för att bland annat hålla lokalerna trivsamma för arbetarna, P\textsubscript{ljus}. För att sedan jämföra datacentrets strömförbrukning med 
fokus på de delar som inte har med servrarnas arbete att göra så divideras alla förbrukningar med P\textsubscript{IT}. Vi får av detta ut ett PUE-värde 
(Power Usage Effectiveness) där det idealiska är att PUE-värdet blir 1.0.
\begin{equation}
    PUE = \frac{P_{IT} + P_{strom} + P_{kylning} + P_{ljus}}{P_{IT}}    
\end{equation}
Kärnan av datacentret är en mängd servrar som i sig är en mängd processenheter (CPU, GPU etc.), minnen och hjälpkomponenter. 
När dessa komponenter arbetar så omvandlas elektrisk energi till värmeenergi. Denna värme skadar komponenterna när den uppnår nog höga temperaturer. Därför behövs  
ett system för kylning för att transportera bort denna värme.
\subsection{Luftkylning}
Det vanligaste sättet att kyla datacenter är via luftkylning. Det går till så att kyld luft 
riktas genom servrarna med hjälp av fläktar~\cite{modelling2,modelling1}. Luften värms upp av de varma servrarna och leds vidare till en anordning som kyler den. 
Därifrån leds sedan den kylda luften tillbaka mot servrarna. Detta görs via fläkt- och tunnelsystem.
\subsection{Flytande kylning}
Ett annat sätt att kyla datacentret är med ett flytande ämne ~\cite{energycompare2}. Vatten är det mest vanliga, men det finns många andra kylmedel så 
som t.ex HFC134a eller HFO1234ze. Vätskan kan via ledningar föras tätt intill de delar i servrarna som skapar mest värme, eller användas för att kyla ned den 
luft som i sin tur genomför den direkta kylningen. Uppvärmd vätska pumpas bort och ny kyld vätska tillförs. En flytande vätska 
blir beroende av pumpar och bibehållet tryck i ledningsystemet. Används ett antistatiskt kylmedel~\cite{immersioncooling1} så kan hela servrarna sänkas ner i ett 
bad av kylmedel. Kylmedel har en påverkan på klimatet, de exempel nämnda ovan har ett GWP-värde (Global Warming Potential)~\cite{GWP}.
\subsection{Tvåfaskylning}
Tvåfaskylning heter så för att det ämne som används för värmetransport ändrar tillstånd från flytande till gas. Detta då ämnets gastemperatur är lägre 
än temperaturen på de varma komponenterna. I gasform transporteras kylmedlet till en kondensator som kyler ner gasen igen och sedan pumpas samma kylmedel tillbaka igen, 
nu i flytande form~\cite{energycompare2}.  
\subsection{Gratis kylning}
Vi kan utnyttja kylan i luften eller vattnet (massa) utanför datacentrets byggnad, genom att ta in massa i vårt system eller låta massan transportera bort värme från vårt slutna 
system ~\cite{free-cooling2}. Samma massa som tas in måste dock kunna avlägnas, men en fördel är att den luft eller vatten som avlägsnas kan återvinnas.
\subsection{Återvunnen energi}
Det går att återanvända värmeenergin, för att få ytterligare effektivitet. Detta kan som i Boden \cite{free-cooling1} innebära att värmen transporteras 
till ett intilligande växthus. Där skapar den en växtzon uppe i kallaste norden, där grönsaker kan odlas året om. 
\begin{equation}
    PUE = \frac{P_{IT} + P_{strom} + P_{kylning} + P_{ljus} - P_{atervinn}}{P_{IT}}
\end{equation}
Vid återvinning av värmeenergin~\cite{modelling2} subtraheras strömåtgången med P\textsubscript{atervinn} i PUE ekvationen. P\textsubscript{atervinn} är den mängden ström 
som skulle behövts användas för att till exempel värma upp växthuset i Boden utan datacentrets värme. Detta skulle teoretiskt sett kunna ge
ett PUE-värde under 1.

\section{Analys}
En fördel med luftkylning är att det är den vanligaste kylningsanläggningen och delvis därför enklare att underhålla~\cite{coolingcompare1}. 
En trasig fläkt är mer normalt för en operatör att byta ut än pumpar, rör och slangkopplingar i ett vätskesystem. Vätskesystemet behöver pågrund av läckrisken mer övervakning 
vilket också blir en kostnad utöver utbildning.

En nackdel med luft är att anläggningen för att kontrollera luftflödet tar upp väldigt mycket plats \cite{modelling2}. Eftersom lufts värmeledningsförmåga 
är 23ggr mindre jämfört med vattnet \cite{thermalconduct} så blir det mycket mer massa som behöver transporteras. Detta gör att transportrummet för luft måste ha större dimensioner än för vatten.
Andra nackdelar med luftkylning är dess relativa höga ljudnivå ~\cite{immersioncooling1}, att den relativa luftfuktigheten behöver hanteras och att dess värmeavledningsförmåga
kan överstigas vid högre koncentration av energi.

För att undvika att behöva kyla ner all luft med element (CRAC) \cite{energy3} som kan bli 40\% av datacentrets strömförbrukning \cite{heatexchange1}, kan luft 
tas in utifrån \cite{free-cooling2}. Detta kräver en utbyggd anläggning för att rengöra den inkommande luften på partiklar och ge den rätt luftfuktighet. 
På detta sätt kan förbrukningen sänkas med 45\%.

Även vatten kan tas in utifrån men då finns det mer regleringar på mängden insamlat vatten som behöver följas \cite{liqcool2}. Vid användande av vattenmagasin kan vatten 
samlas när den är tillgänglig eller när det är billigare att kyla den. Sen kan vi använda det kylda vattnet när elen är dyrare och på detta sätt
få ner våra kostnader för kylning.

När de kommer till vätskekylning så kan en högre precision på kylningen nås \cite{energycompare2} genom att föra vätskorna nära komponenterna. Om vatten används så finns en risk  
för läckor som är skadliga för de elektriska komponenterna \cite{liqcool1}. Därför är det en fördel att använda sig av antistatiska vätskor ifall kylningen är komponentnära. 
Riskerna sänks och därmed underhållskostnaden. Men läckor ska undvikas även med kylmedel eftersom dessa vätskor  generellt bidrar till växthuseffekten. Flera av de kylmedel 
som används har ett GWP-värde, som t.ex HFC134a på 1470 \cite{GWP1}. Vilket gjort att de blir förbjudna i framtiden inom både EU~\cite{EUphase}. Därför är det viktigt att välja rätt. 

Pågrund av kylmedlens höga precision och lägre beroende av yta så är de ett bra komplement på högpresterande enheter. Då mycket utveckling just nu sker för att 
få fram de snabbaste AI kretsarna så kan energikoncentrationen öka snabbt i närtid, speciellt då datacentret vill erbjuda den senaste tekniken med fokus på prestanda inom AI. 

Tvåfaskylning är en annan nisch av kylmedel. En fördel med tvåfaskylning är att det i gasform transporterar sig på självtryck. Vi får därför en lägre 
strömkostnad för pumpar vilket är en betydlig del i vätskesystem. En tvåfaskylning kan kombineras med en lösning där serverutrustningen är nedsänkt i 
vätskan. Detta kräver stora kostnader \cite{liqcool2} i ombyggnad av datacenter men vid högpresterande servrar kan det vara motiverat. 
En nackdel för nedsänkt kylning är att underhållet kan också bli högre då operatörer behöver utbildas och ett tvåfassystem 
behöver vara slutet för att samla in gasformen. Då behöver servrarna antagligen gå ner i värmeutveckling vid underhåll som kräver brytning av förslutningen.  

När det kommer till att återvinna energin har vi sett exempel på att energin använts för att värma upp växthus på kalla platser. Med det i åtanke kan tänkas att 
intilliggande byggnaders \cite{reuse1} värmesystem kan kopplas ihop. Fördelen med vatten är att den är en bättre tranportör av värme. 
Dessutom så används vatten i våra byggnaders värmesystem redan vilket gör det till ett bra element att koppla ihop med datacenters vätskesystem. Därför blir vattnet i kombination 
med gratis luftkylning en effektiv energilösningen. För att minska underhållskostnad i oro för vattenläcka så bör vattenkylda dörrar som monteras på serverrackarna 
nyttjas som den varma luften får transporteras genom för att värma upp vattnet. På detta sätt har en superdator på KTH värmt upp intillliggande labbsalar~\cite{reuse1}.  
\section{Slutsats}
Denna artikel har redovisat kylsystem för luftkylning, flytande kylning och tvåfaskylning. Matematiska formler för att visa på hur effektivt datacentret använder strömmen enligt 
PUE har presenterats och genom att titta på gratis kylning kunnat se att PUE-värdet kan sänkas drastiskt. Utnyttjandet av gratis kylning kräver ytterligae infrastrukur för att ta 
tillvara på denna resurs. För att ytterligare sänka data centrets energiavtryck kan återvinningsmöjligheter av energin planeras in genom att värma upp intilliggande byggnader. 
Denna artikel ger läsaren en bättre förståelse för hur ett datacenter kan kylas och att framtiden kräver en mer effektivt utnyttjande av energin där gratis kylning utnyttjas och 
vätskekylningens energi återvinns. 

\printbibliography
\end{document}